\documentclass{cshonours}
\usepackage{url}
\usepackage{graphics}
\usepackage{bibunits}
\usepackage{abbrevs}
\usepackage{acronym}
\usepackage[vario]{fancyref}
\usepackage{gnuplot-lua-tikz}
\usepackage{xspace}

%%% BEGIN LATEX TWEAKS

% Configure bibliography
\bibliographystyle{acm}
\defaultbibliography{../references/primary}
\defaultbibliographystyle{acm}

% Acronyms for common stuff
\newcommand{\acrodefn}[3]{%
	\acrodef{#1}[#2]{#3}%
	\expandafter\newcommand\csname#1\endcsname{\ac{#1}\xspace}%
}
\acrodefn{aecl}{AECL}{Atomic Energy Canada Limited}
\acrodefn{cgr}{CGR}{Compagnie General Radiographique}
\acrodefn{fda}{FDA}{U.S. Food and Drug Administration}
\acrodefn{maude}{MAUDE}{Manufacturer and User Facility Device Experience Database}

% Abbreviation commands for common stuff
% TODO: Fix spacing problem
\newabbrev\ther{Therac-25}
\newabbrev\etal{et al.}

% Fancyref support for subsections, source; https://github.com/openlilylib/tutorials/blob/master/aGervasoni/orchestralScores/example-materials/OLLbase.sty
\newcommand*{\fancyrefsubseclabelprefix}{subsec}

\fancyrefaddcaptions{english}{%
  \providecommand*{\frefsubsecname}{subsection}%
  \providecommand*{\Frefsubsecname}{Subsection}%
}

\frefformat{plain}{\fancyrefsubseclabelprefix}{\frefsubsecname\fancyrefdefaultspacing#1}
\Frefformat{plain}{\fancyrefsubseclabelprefix}{\Frefsubsecname\fancyrefdefaultspacing#1}

\frefformat{vario}{\fancyrefsubseclabelprefix}{%
  \frefsubsecname\fancyrefdefaultspacing#1#3%
}
\Frefformat{vario}{\fancyrefsubseclabelprefix}{%
  \Frefsubsecname\fancyrefdefaultspacing#1#3%
}


%%% END LATEX TWEAKS


\title{Therac-25: Will history repeat itself?}
\author{Ash Tyndall}

\begin{document}
\maketitle

\tableofcontents

\chapter{Introduction}

In the early 1970s, two companies, \aecl and \cgr, collaborated to build updated models of their core radiotheraputic offerings; Medical Linear Accelerators (LINACs). LINACs are an extension of the basic concept of a linear particle accelerator, repackaged for medical applications. They are are designed to create a beam of either electron or x-rays which can be focused on a very specific section of a patients body. These beams can be used to kill cancerous growths without severely damaging surrounding tissue.

Together, \aecl and \cgr develop two LINACs; the Therac-6 and the Therac-20, both of which were based on previous \cgr work, but with mechanical control substituted with control via computer terminal. These machines are designed with accelerators that could produce 6 MeV x-rays and 20 MeV x-rays/electrons respectively. Still in partnership in the mid-1970s, \aecl and \cgr develop a new Therac that uses a new kind of linear accelerator, a ``double-pass'' system, which reduces the space requirements and allows cheaper parts to be used. \aecl and \cgr part ways soon thereafter, citing competitive pressures. 

\aecl proceeded with the development of this new ``double-pass'' system, the \ther (so named for its 25 MeV x-ray and electron beam), continuing to develop the software-based control system, and passing the necessary regulatory requirements. The system is announced for public sale in 1983. One of the cutting-edge features of the machine is the removal of hardware-based interlocks and safeguards on the device, \aecl instead opting to use a wholly software-based approach to ensure that the appropriate components are rotated in front of the raw electron beam to reduce the dangerous radiation to therapeutic levels.

In 1985, the first reported case of \ther software safeguard failure occurred. Katherine Yarbrough, a breast cancer patient, receives an estimated 15,000--20,000 rads instead of the normal 200 rad range. She suffers severe radiation burns and shoulder and arm paralysis. A month later at a different facility, an unidentified female patient receives four separate overdoses totalling 13,000--17,000 rads within the space of several minutes, due to a combination of safeguard failure and poorly explained error messages. This patient dies of radiation induced cancer some months later.

It was initially unclear to those who operated the \ther that software errors were the cause of these overdoses. Due to the nature of radiation overdoses, the most serious of symptoms appear days if not weeks later, causing the resulting deaths and disablements to be attributed to other factors. However, over time, it became clear to different system operators that something was seriously wrong with the \ther, sparking an eventual forced FDA recall of the product with a total of six overdoses and two deaths.

% TODO: Cite
\ther was responsible for the first known deaths in radiotherapy, a profession that began some 35 years prior, and the confluence of death and computing caused an uproar at the time. Since then, the flames of anger have died down, and there is an opportunity to review the incident objectively. This report will examine the \ther case in detail, trying to determine the answers to several important questions:
\begin{enumerate}
 \item How does one design safe software, what does it involve, what are the pitfalls and how do we avoid them? (\Fref{chap:safesoftware})
 \item What were the flaws of \ther and how did failures on the part of \aecl and \cgr contribute to the creation of these flaws? (\Fref{chap:flawsfailures})
 \item How did standards and regulatory bodies respond to \ther through new guidelines and processes for creation of ``safe'' medical device software? (\Fref{chap:newstandards})
 \item Have those guidelines and processes resulted in the creation of safer medical device software? (\Fref{chap:safer})
 \item Are there still areas in the medical landscape where regulation is insufficient? (\Fref{chap:reggaps})
 \item Will \ther happen again? (\Fref{chap:conclusion})
\end{enumerate}

% TODO: Economic consequences of human death

\chapter{Literature Review}
\label{chap:litreview}
\ther was one of the first widely reported software-related disasters, and as a result, a variety of academic work has been performed since the disaster investigating the specifics of the failure of \ther, as well as the development of safety critical  medical software generally.

Primary work in the area of safety critical software development and system design as been done by Nancy Leveson. Her two books \textit{Safeware} \cite{safeware} and the more recent \textit{Engineering a Safer World} \cite{saferworld} are highly influential works in the field. They discuss from a variety of disciplines the causes of safety-critical system failure, from poor software testing processes to a failure of system design to adequately inform the user of the system's state. Of particular relevance to our \ther questions is \textit{Safeware} chapter 6, ``The Role of Humans in Automated Systems'', which describes several models of human-computer interaction and the necessary design principles to properly enable them. Additionally, \textit{Engineering a Safer World} chapter 9, ``Safety-Guided Design'', which discusses several of the \ther design flaws.

Leveson and Turner have also contributed the primary academic report on \ther \cite{leveson1993investigation} which discusses the history of the Therac brand, the history of the companies involved, as well as the timeline of the disasters that ensued.

Various work has been done into the area of safe critical software from both the perspective of cause and prevention. Dunn \cite{dunn2003designing} provides us with a helpful definition of safety in terms of ``mishap risk'', as well as examples of mishap causes.

In terms of cause, in \textit{Failure in Safety-Critical Systems} chapter 3 \cite{johnson2003failure} Johnson discusses in detail the sources of failure, touching upon a broad variety of failures including Regulatory, Managerial, Hardware, Software, Human and Team based failures. Besnard and Baxter \etal \cite{besnard2003human} discuss two models of system failure, Reason's swiss cheese model and Randell's fault-error-failure model, both of which can be used to analyse the \ther disaster.

In terms of prevention, Nolan \cite{nolan2000system} proposes several strategies to be considered when designing ``safe systems of care'', as well as methods to reduce ``adverse events''  which may compromise patient health. Obradovich and Woods \cite{obradovich1996users} perform an investigation of poor Human-Computer Interface design in a medical device, describing how the device is flawed and how both the user and medical supervisor can change their processes to cope with this. Lin, Vicente and Doyle \cite{lin2001patient} propose a new interface for a specific medical appliance that applies human factors engineering, a key part of the discussed prevention of user error, and provides data demonstrating the effectiveness of such an approach from error minimisation and efficiency perspectives.

Several works discuss the introduction of various international standards and regulations post-\ther. Rakitin \cite{rakitin2006coping} provides an overview of foundational standards in the risk management space of medical device software. Brown \cite{brown2000overview} provides an overview of specifically the IEC 61508 ``Design of electrical / electronic / programmable electronic safety-related systems'' standard. Jordan \cite{jordan2006standard} provides an overview of specifically the IEC 62304 ``Medical Device Software -- Software Lifecycle Processes'' standard.

To allow us to determine if regulation of medical device software has improved the situation, several works which analyse relevant data will be examined. Wallace and Kuhn produced two papers \cite{wallace1999lessons,wallace2001failure} analysing a subset of \fda data relating to ``adverse events'' and provided statistics on the types of software errors and the medical domain of the devices, as well as information on how to prevent and detect these types of errors. The \fda \maude dataset \cite{maude} is also examined directly by the author to attempt to derive conclusions regarding the proportion of medical device software errors in the broader \maude database. Finally, \textit{Failure in Safety-Critical Systems} chapter 5 \cite{johnson2003failure} discusses the under-reporting of incidents, as well as reporting bias.

\chapter{Safe software design: What does it involve?}
\label{chap:safesoftware}

\chapter{Flaws and failures: How and why?}
\label{chap:flawsfailures}

\chapter{Regulatory response: New standards}
\label{chap:newstandards}

\chapter{Data analysis: Are we safer?}
\label{chap:data}

\chapter{Regulatory gaps: What's next?}
\label{chap:reggaps}

\chapter{Conclusion: Will history repeat?}
\label{chap:conclusion}


\appendix

\bibliography{primary}

\end{document}